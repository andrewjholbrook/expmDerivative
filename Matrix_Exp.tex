\documentclass[twoside]{article}
\usepackage[utf8]{inputenc}

\usepackage[margin=1in]{geometry}

\usepackage[section]{placeins}

\title{Some Remarks on Computing Gradients of a Matrix Exponential}
\date{\today}
%\author{Nathan E. Glatt-Holtz, Andrew J. Holbrook\\
%\scriptsize{emails: negh@tulane.edu, aholbroo@g.ucla.edu}}

\usepackage{amsfonts, amssymb, amsmath, amsthm,esint,multicol}
\usepackage[colorlinks=true, pdfstartview=FitV, linkcolor=blue,
            citecolor=blue, urlcolor=blue]{hyperref}
\usepackage[usenames,dvipsnames,table]{xcolor}
\definecolor{Red}{rgb}{0.7,0,0.1}
\definecolor{Green}{rgb}{0,0.7,0}
\usepackage{accents}
\usepackage{comment}
\usepackage{graphicx}
\usepackage[capitalize,nameinlink,noabbrev]{cleveref}
\usepackage{enumerate}
\usepackage{enumitem}
\usepackage{natbib}


\providecommand*\showkeyslabelformat[1]{{\normalfont \tiny#1}}
\usepackage[notref,notcite,color]{showkeys}
\definecolor{labelkey}{rgb}{0,0,1}
\usepackage{bbm}
\usepackage{mathtools} %for coloneqq



\usepackage{tabularx}


% \usepackage{enumitem}
\usepackage{listings}
\usepackage[textsize=tiny]{todonotes}
\usepackage{tikz}
\usetikzlibrary{shapes.misc}
\usepackage{etoolbox}
\usepackage{appendix}
\usepackage{subcaption}
\usepackage{wrapfig}
\usepackage{xr}
\usepackage{booktabs}



\pagestyle{myheadings}
\numberwithin{equation}{section}


\newtheorem{Theorem}{Theorem}[section]
\newtheorem{Proposition}[Theorem]{Proposition}
\newtheorem{Lemma}[Theorem]{Lemma}
\newtheorem{Corollary}[Theorem]{Corollary}
\newtheorem{Example}[Theorem]{Example}
\newtheorem{Definition}[Theorem]{Definition}
\newtheorem{Remark}[Theorem]{Remark}
\newtheorem{Question}[Theorem]{Question}
\newtheorem{Assumption}[Theorem]{Assumption}

\usepackage[section]{algorithm}      %for algorithms
\usepackage[noend]{algpseudocode}    %for algorithms

\newtheorem{Not}{Notation}[section]
\newtheorem*{LemNoNum}{Lemma}

%%%%%% MACROS %%%%%%%%%%

%Number Systems
\newcommand{\RR}{\mathbb{R}}
\newcommand{\CC}{\mathbb{C}}
\newcommand{\NN}{\mathbb{N}}
\newcommand{\ZZ}{\mathbb{Z}}


\newcommand{\tr}{\mbox{Tr}}
\newcommand{\bfU}{\mathbf{u}}
\newcommand{\bfV}{\mathbf{v}}









\begin{document}


\maketitle



%
%\begin{abstract}
%
%
%\end{abstract}

%{\noindent \small {\it {\bf Keywords: }  Bias Estimation,  Metropolis-Hastings Kernel, Parallel (Multiproposal) MCMC.}} \\
%
%\begin{footnotesize}
%\setcounter{tocdepth}{1}
%\tableofcontents
%\end{footnotesize}



%\newpage


\section{The Dynamical Systems View}
We are interested in computing 
\begin{align}
\label{eq:grad:def}
	\nabla_J \exp(t Q) = \lim_{\epsilon \to 0} \frac{\exp(t (Q+ \epsilon J) - \exp(tQ) }{\epsilon}
\end{align}
matrices  $J, Q \in \RR^{d\times d}$.
Noting that $X(t) := \exp(t Q)$ obeys the (matrix valued) ordinary differential equation
\begin{align}
	\label{eq:exptQ}
	\frac{dX}{dt} = Q X, \quad X(0) = I,
\end{align}
and taking $X^\epsilon =  \exp(t Q+ \epsilon J)$, $Y^\epsilon = \epsilon^{-1} ( X^\epsilon- X)$
we have 
\begin{align*}
	\frac{dY^\epsilon}{dt}  = Q Y^\epsilon + J X^\epsilon, \quad Y^\epsilon(0) = 0.
\end{align*}
Hence taking a limit as $\epsilon \to 0$ we find that $Y = \nabla_J \exp(t Q)$ obeys
\begin{align}
	\frac{dY}{dt}  = Q Y + J X, \quad Y^\epsilon(0) = 0,
	\label{eq:sen:eq}
\end{align}
and so variation of constants yields that, for any $t \geq 0$, 
\begin{align}
	Y(t) &= \exp(t Q) \int_0^t \exp(-sQ) J \exp(sQ) ds
	= \exp(t Q) \sum_{k,m = 0}^\infty \int_0^t \frac{(-s)^{k} Q^k J s^{m}Q^m}{k!m!}ds
		\label{eq:var:const:1}\\
	&=  \exp(t Q) \sum_{k,m = 0}^\infty (-1)^k \frac{Q^k J Q^m}{m! k!(m+k+1)} t^{k+m+1}
	=  \exp(t Q) \sum_{k,m = 0}^\infty (-1)^k \frac{(m+k)!}{m! k!} \frac{Q^k J Q^m}{(m+k+1)!} t^{k+m+1}
	\notag\\
	&=  \exp(t Q) \sum_{n = 0}^\infty \frac{t^{n+1}}{(n+1)!}  
	\left( \sum_{l = 0}^n (-1)^l \binom{n}{l} Q^l J Q^{n -l}\right)
	\label{eq:var:const:2}
\end{align}
Note that the last line corresponds to the identity from \cite{najfeld1995derivatives}.  Taking the  first order
approximation in \eqref{eq:var:const:2} gives
\begin{align}
	\tilde{\nabla}_J \exp(t Q) = t  \exp(t Q) J,
\end{align}
which we see from \eqref{eq:var:const:1} is evidently exact in the case when $J$ and $Q$ commute. 



Next notice that if we differentiate $\tilde{Y} =  t  \exp(t Q) J$ we find that $\tilde{Y}$ obeys
\begin{align}
	\frac{d \tilde{Y}}{dt} = Q \tilde{Y} +  \exp(t Q) J, \quad \tilde{Y}(0) = 0.
\end{align}
Taking 
\begin{align}
Z(t) = Y(t) - \tilde{Y}(t) = \nabla_J \exp(t Q) - t  \exp(t Q) J
\label{eq:true:diff}
\end{align}
and comparing with \eqref{eq:sen:eq} yields
\begin{align}
\label{eq:Z:dym}
	\frac{dZ}{dt} = Q Z + J \exp(t Q) -  \exp(t Q) J, \quad Z(0) = 0.
\end{align}
and hence
\begin{align}
	\label{eq:ass:procs}
	Z(t) = \exp(t Q) \int_0^t (\exp(-sQ) J \exp(s Q) - J) ds
	=  \exp(t Q) \sum_{n = 1}^\infty \frac{t^{n+1}}{(n+1)!}  
	\left( \sum_{l = 0}^n (-1)^l \binom{n}{l} Q^l J Q^{n -l}\right)
\end{align}
as could also be directly deduced from \eqref{eq:var:const:1}, \eqref{eq:var:const:2}.


\section{Time Asymptotic Approximation Error}

We would like to deduce the time asymptotic behavior of $Z(t) := \nabla_J \exp(t Q) - t  \exp(t Q) J$.    Evidently we need some further
structure for $Q$.   Consider the class of (irreducible) rate matrices 
\begin{align}
	\mathcal{R} = \{ Q \in \RR^{d\times d} | Q_{i,j} > 0 \text{ for } j \not= i \text{ and } \sum_{j =1}^d Q_{i,j} = 0
	\text{ for each } i\}
\end{align}
and subclass of diagonalizable matrices 
\begin{align}
	\mathcal{D}  =  \{ Q \in \mathcal{R} |& Q \text{ is diagonalizable and for a suitable diagonalizing matrix } M, D = M Q M^{-1} \notag\\
		& \Re D_{i,i} < 0, \text{ for } i = 1, \ldots, d-1, D_{d,d} = 0\}.
		\label{eq:Q:spc}
\end{align}
Notice that, for $Q \in \mathcal{D}$, 
\begin{align*}
Q \exp(t Q) = M^{-1} D M M^{-1} \exp(t D) M = M^{-1} D \exp(t D) M
\end{align*}
 and hence
\begin{align}
	\| Q e^{tQ} \| \leq  \| M^{-1}\| \|M\| \left( \sum_{j =1}^{d-1} | D_{i,i} \exp( t D_{i,i}) |^2 \right)^{1/2}
	\leq  \sqrt{d-1}\| M^{-1}\| \|M\| e_m \exp(-t e_M)
	\label{eq:decay:est:1}
\end{align}
where here and below $\| \cdot \|$ denotes the standard Frobenius norm on $\CC^{d \times d}$
and 
\begin{align}
	e_m = \min_{i = 1, \ldots, d-1} |D_{i,i}|, \quad e_M = \max_{i = 1, \ldots, d-1} |D_{i,i}|.
	\label{eq:max:min:D}
\end{align}
Thus $\| Q e^{tQ} \|$ decays with an exponential rate and so with \eqref{eq:exptQ} so does
$de^{-tQ} /dt$

Building on this observation and \eqref{eq:ass:procs} we might expect, for $t$ large, that
\begin{align}
	Z(t) \approx& \exp(t Q) J  \sum_{n = 1}^\infty \frac{t^{n+1}Q^n}{(n+1)!}  
	= \exp(t Q) J Q^+  \sum_{n = 1}^\infty \frac{t^{n+1}Q^{n+1}}{(n+1)!}  
	= \exp(t Q) J Q^+  (\exp(tQ) - I - t Q)\\
	\approx& -\exp(t Q) J Q^+  ( I + t Q)
	\label{eq:ass:procs:der}
\end{align}
where $Q^+$ is the pseudo-inverse $Q$. 
Notice that
\begin{align}
 \tilde{Z}(t) := -\exp(t Q) J Q^+  ( I + t Q)
 \label{eq:t:ass:diff}
\end{align}
obeys 
\begin{align}
\label{eq:tZ:dym}
	\frac{d \tilde{Z}}{dt} = Q \tilde{Z} -\exp(t Q) J Q^+Q, \quad \tilde{Z}(0) =- J Q^+.
\end{align}
Here in our heuristic \eqref{eq:ass:procs:der} we used our structural assumptions on $Q \in \mathcal{D}$ 
to infer that 
\begin{align*}
Q = Q Q^+ Q = Q^+ Q^2.
\end{align*}
To justify dropping $\exp(t Q) J Q^+ e^{tQ}$
as small for large $t$ we observe analogously to \eqref{eq:decay:est:1}
\begin{align}
	\| Q^+ e^{tQ} \| \leq  \| M^{-1}\| \|M\| \left( \sum_{j =1}^{d-1} | D_{i,i}^{-1} \exp( t D_{i,i}) |^2 \right)^{1/2}
	\leq  \sqrt{d-1}\| M^{-1}\| \|M\| e_M^{-1} \exp(-t e_M).
	\label{eq:decay:est:2}
\end{align}
with $e_M$ defined as in \eqref{eq:max:min:D}.



Let us now prove that indeed  $ -\exp(t Q) J Q^+  ( I + t Q) \approx \nabla_J \exp(t Q) - t  \exp(t Q) J$ with an error decaying at an exponential rate as $t \to \infty$.  
More precisely we have the following
\begin{Theorem}\label{thm:ass:error}
For any $Q \in \mathcal{D}$ and any $J \in \CC^{d \times d}$
\begin{align}
   \| -\exp(t Q) J Q^+  ( I + t Q) + Q^+ J (I- Q Q^+)  - (\nabla_J \exp(t Q) - t  \exp(t Q) J)\| \leq Ce^{-\kappa t}
\end{align}
for constants $C, \kappa > 0$ depending on $Q$ and $J$ explicitly as
\begin{align}
	C :=  \quad
	\kappa := 
\end{align}
\end{Theorem}
\begin{proof}
Take $W(t) =  \tilde{Z}(t) - Z(t)$ where are defined as in \eqref{eq:t:ass:diff}, \eqref{eq:true:diff} respectively.
Notice that $W$ follows the dynamic, cf. \eqref{eq:Z:dym}, \eqref{eq:tZ:dym}
\begin{align}
 	\frac{d W}{dt} = Q W +  \exp(t Q) J (I - Q^+Q) -  J \exp(t Q) \quad W(0) =- J Q^+
	\label{eq:ass:err:eq}
\end{align}
With variation of constants we arrive at the expression
\begin{align}
W(t) = - \exp(tQ)\left(J Q^+ + \int_0^t (\exp(-sQ) J \exp(s Q) - J (I - Q^+Q)) ds\right)
	\label{eq:ass:err:sol}
\end{align}
But we observe the following for expression inside of the integral 
\begin{align*}
	\exp(-sQ)& J \exp(s Q) - J (I - Q^+Q)\\
	=& \exp(-sQ) Q Q^+ J \exp(s Q) + \exp(-sQ) (I - Q Q^+ )J \exp(s Q) - J (I - Q^+Q)\\
	=&\exp(-sQ) Q Q^+ J \exp(s Q) + (I - Q Q^+ )J \exp(s Q) - J (I - Q^+Q)\\
 	=&\exp(-sQ) Q Q^+ J \exp(s Q) + (I - Q Q^+ )J QQ^+ \exp(s Q) \\
	    &+ (I - Q Q^+ )J( I - QQ^+) \exp(s Q) - J (I - Q^+Q))\\
	=&\exp(-sQ) Q Q^+ J \exp(s Q) + (I - Q Q^+ )J QQ^+ \exp(s Q) \\
	    &+ (I - Q Q^+ )J( I - QQ^+)  - J (I - Q^+Q))\\
	=&  \exp(-sQ) Q Q^+ J \exp(s Q) + (I - Q Q^+ )J QQ^+ \exp(s Q) - Q Q^+ J( I - QQ^+) \\
	=&  \exp(-sQ) Q Q^+ J  Q Q^+ \exp(s Q) +  \exp(-sQ) Q Q^+ J (I- Q Q^+)  \\
	  &+ (I - Q Q^+ )J QQ^+ \exp(s Q) - Q Q^+ J( I - QQ^+) 
\end{align*}
In deriving the above identity we have repeatedly used that 
\begin{align*}
	(I - Q Q^+ ) \exp(s Q)  
	&= M^{-1} M (I - Q Q^+ ) \exp(s Q) M^{-1} M
	= M^{-1} M (I - Q Q^+ )M^{-1} M \exp(s Q) M^{-1} M\\
	&= M^{-1} (I - D D^{+})   \exp(s D) M = I - Q Q^+
\end{align*}
Hence combining the above we find
\begin{align*}
	W(t)% =& - \exp(tQ)\biggl(J Q^+ + \int_0^t (\exp(-sQ) Q Q^+ J \exp(s Q) + (I - Q Q^+ )J QQ^+ \exp(s Q) - Q Q^+ J( I - QQ^+) ds\biggr)\\
	       =& -\exp(tQ)\biggl(J Q^+ +  \int_0^t (I - Q Q^+ )J QQ^+ \exp(s Q) + \exp(-sQ) Q Q^+ J (I- Q Q^+) ) ds\biggr)\\
	          & - \int_0^t \exp((t-s)Q)  Q^+ Q J Q^+ Q\exp(s Q)ds +t \exp(tQ) Q^+ Q J( I - QQ^+) \\
	        :=& T_1(t) + T_2(t) + T_3(t)
\end{align*}
%\begin{align*}
%	W(t) =& - \exp(tQ)\left(J Q^+ + \int_0^t (\exp(-sQ) J \exp(s Q) - J (I - Q^+Q)) ds\right)\\
%	=& - \exp(tQ)\left(J Q^+ + \int_0^t (\exp(-sQ) Q Q^+ J \exp(s Q) + \exp(-sQ) (I - Q Q^+ )J \exp(s Q) - J (I - Q^+Q)) ds\right)\\
%	=&- \exp(tQ)\left(J Q^+ + \int_0^t (\exp(-sQ) Q Q^+ J \exp(s Q) + (I - Q Q^+ )J \exp(s Q) - J (I - Q^+Q)) ds\right)\\
%	=&- \exp(tQ)\biggl(J Q^+ + \int_0^t (\exp(-sQ) Q Q^+ J \exp(s Q) + (I - Q Q^+ )J QQ^+ \exp(s Q) \\
%		                          &\qquad \qquad \qquad \qquad + (I - Q Q^+ )J( I - QQ^+) \exp(s Q) - J (I - Q^+Q)) ds\biggr)\\
%	=&- \exp(tQ)\biggl(J Q^+ + \int_0^t (\exp(-sQ) Q Q^+ J \exp(s Q) + (I - Q Q^+ )J QQ^+ \exp(s Q) - Q Q^+ J( I - QQ^+) ds\biggr)
%\end{align*}
Regarding $T_1(t)$ observe that
\begin{align*}
T_1(t)= -\exp(tQ)&\biggl(J Q^+ +  \int_0^t (I - Q Q^+ )J QQ^+ \exp(s Q)  +\exp(-sQ) Q Q^+ J (I- Q Q^+))ds\biggr) \\
=& -\exp(tQ)\biggl(J Q^+ +  \int_0^t \frac{d}{ds}[ (I - Q Q^+ )J Q^+ \exp(s Q)  - \exp(-sQ) Q^+ J (I- Q Q^+) ]ds\biggr)\\
=& -\exp(tQ)\biggl(J Q^+ + (I - Q Q^+ )J Q^+\exp(tQ) - \exp(-tQ) Q^+ J (I- Q Q^+)  \\
      &\qquad \qquad  \quad - (I - Q Q^+ )J Q^+
     +  Q^+ J (I- Q Q^+) \biggr)\\
&= -\exp(tQ)\biggl(Q Q^+J Q^+ + (I - Q Q^+ )J Q^+\exp(tQ)  +Q^+ J (I- Q Q^+)  \biggr)
     - Q^+ J (I- Q Q^+)
\end{align*}
Turning to $T_2(t)$ and letting $\bar{J}=M^{-1} J M$ and $I_{-d}$ be the identity matrix with final diagonal element set to 0, we have
\begin{align*}
	 T_2(t)  &=  \int_0^t \exp((t-s)Q)  Q^+ Q J Q^+ Q\exp(s Q)ds  \\
	&= M  \left(\int_0^t (\exp((t-s)D) I_{-d} )  \bar{J} (I_{-d} \exp(s D))ds \right) M^{-1} \\
	&= M  \left(\int_0^t   \bar{J} \circ \Psi(t,s)  ds \right) M^{-1}  = M  \left(\bar{J} \circ \int_0^t    \Psi(t,s)  ds \right) M^{-1} \\
	&= M  \left(\bar{J} \circ    \Phi(t)   \right) M^{-1} \, 
\end{align*}
where the elements of $\Psi(t,s)$ satisfy
\begin{align*}
	\Psi_{ij}(t,s) = \Bigg \{ \begin{array}{cc}
		e^{(t-s) D_{j,j} + sD_{i,i}}  \, , & i,j < d ,\, i\neq j \\
		e^{tD_{j,j}} \, , &  i,j < d ,\, i= j  \\
		0          \, , & o/w  
	\end{array} \, ,
\end{align*}
and
\begin{align*}
 	\Phi_{ij}(t) = \int^t_0 \Psi_{ij}(t,s) ds = \Bigg\{ \begin{array}{cc}
 		\frac{e^{tD_{i,i}}-e^{tD_{j,j}}}{D_{i,i}-D_{j,j}}  \, , & i,j < d ,\, i\neq j \\
 		te^{tD_{j,j}} \, , &  i,j < d ,\, i= j  \\
 		0          \, , & o/w  
 	\end{array}   \, .
\end{align*}
Then, by the definition of the Frobenius norm
\begin{align*}
	\lVert T_2(t)  \rVert &= \lVert M  \left(\bar{J} \circ    \Phi(t)   \right) M^{-1}  \rVert  \leq  \lVert \bar{J} \circ    \Phi(t) \rVert    \lVert M \rVert \lVert M^{-1}  \rVert  \\
	&=  \left(\sum_{i,j} |\bar{J}_{ij}|^{2} |\Phi_{ij}(t) |^2 \right)^{1/2}  \lVert M \rVert \lVert M^{-1}  \rVert\\
	&\leq \left(\sum_{i,j} |\Phi_{ij}(t)|^4 \right)^{1/4} \left( \sum_{i,j} |\bar{J}_{ij}|^{4} \right)^{1/4}   \lVert M \rVert \lVert M^{-1}  \rVert \\
	&\leq \left( (D-1)^2 \max_{i,j} |\Phi_{ij}(t)|^4 \right)^{1/4}   \lVert \bar{J}^{\circ 2} \rVert^{1/2}   \lVert M \rVert \lVert M^{-1}  \rVert  \\
	&= \left( \max_{i,j} |\Phi_{ij}(t)| \right)\sqrt{D-1} \lVert \bar{J}^{\circ 2} \rVert^{1/2}   \lVert M \rVert \lVert M^{-1}  \rVert  \, .
\end{align*}
Since $\max |\Phi_{ij}|$ goes to 0 at an exponential rate, we have $\lVert T_2(t)\rVert \rightarrow 0$ as $t\rightarrow \infty$.  Finally,
\begin{align*}
	\lVert T_3(t) \rVert &= \lVert t \exp(tQ) Q^+ Q J( I - QQ^+)  \rVert \\
	&\leq \left(t \max_{i<d} e^{tD_{i,i}} \right) \lVert J \rVert  \rVert \lVert M \rVert^2 \lVert M^{-1} \rVert^2 \, .
\end{align*}
 


%as $\| A\| = \sqrt{\tr(AA^*)}$ and some elementary properties of the trace operator
%\begin{align*}
%   \| \exp(-sQ)   J \exp(s Q)\| 
%   =& \sqrt{\tr( \exp(-sQ)   J \exp(s Q) \exp(sQ^*)   J^* \exp(-s Q^*) )}\\
%      =&\sqrt{\tr(    J \exp(s Q) \exp(sQ^*)  J^* \exp(-s Q^*) \exp(-sQ) )}
%\end{align*}
[FINISH]
%and hence with \eqref{eq:decay:est:2} we find
%\begin{align}
%\|T_1(t)\| \leq 2  \sqrt{d-1}\| M^{-1}\| \|M\| ( \|J \| \|Q^+\| )e_M^{-1} \exp(-t e_M)
%\end{align}
%Combining these identities we obtain
%\begin{align*}
% W(t) = \exp(tQ)\left(Q^+Q J Q^+ + (I - Q Q^+ )J Q^+\exp(tQ)
% -Q^+ \int_0^t (\exp(-sQ) Q  J \exp(s Q) - QJ( I - QQ^+) ds \right)
%\end{align*}
%Thus the desired result follows from the fact that 
%$Q^+ \exp(tQ)$ decays with an exponential rate and that 
%$\sup_{t \geq 0} \| \exp(-sQ) Q  J \exp(s Q)\| < \infty$.
\end{proof}

\section{High-dimensional Asymptotics}


Given the above result, it would be interesting to characterize the behavior of 
\begin{align*}
\lVert	Q^+ J (I-QQ^+)  \rVert &= \lVert M^{-1}D^+M J M^{-1} (I-I_{-d}) M \rVert \\
&\leq \lVert D^+ \rVert \lVert M\rVert^2 \lVert M^{-1}\rVert^2 \lVert J \rVert  \\
&\leq \frac{1}{e_m}\lVert M\rVert^2 \lVert M^{-1}\rVert^2 \lVert J \rVert  
\end{align*}
for random $Q$ as $d\rightarrow \infty$.  We are therefore interested in the complex modulus of the non-zero eigenvalue closest to the origin on the complex plain. 

Questions/Comments:
\begin{itemize}
	\item Empirically, $\lVert I-Q^+Q\rVert$ appears to go to 1 as $d$ gets large.   In which case, one can remove the exponents from $\lVert M \rVert$ and $\lVert M^{-1} \rVert$.
	\item Empirically, $e_m \rightarrow \infty$ as $d \rightarrow \infty$.
\end{itemize}

\section{Questions, Comments and Clarifications}

\begin{itemize}
\item[(i)] Where does the extra correction term $-Q^+ J (I- Q Q^+)$ come from?
\item[(ii)] $\mathcal{D}$ defined as \eqref{eq:Q:spc} where $Q$ lives is not a linear space?  How does HMC work here?  What is the underlying target distribution on $\mathcal{D}$?
\item[(iii)] How does the surrogate trajectory method implied by $\tilde{\nabla}_J \exp(t Q) = t  \exp(t Q) J$ perform relative to 
e.g. an adjoint method which exactly computes $\nabla_J \exp(t Q)$?
\item[(iv)] Why not use a `higher order' approximation of $\nabla_J \exp(t Q)$?  For example we could take
\begin{align}
	\nabla_J \exp(t Q) \approx \exp(t Q) \int_0^t (I - sQ) J (I + sQ)ds = \exp(tQ) \left(tJ + \frac{t^2}{2} (JQ - QJ) -\frac{t^3}{3} QJ Q\right).
\end{align}
Why is this a big deal to compute?
\item[(v)] Maybe we should try to prove a higher order version of \eqref{thm:ass:error} in any case.
\item[(vi)] The formula \eqref{eq:var:const:2}, \eqref{eq:var:const:1} may be useful for developing surrogate trajectory methods for other Matrix coefficient estimation problems.
\end{itemize}

\subsection{Estimation Problem 1: Determining the `Potential' in a Toy QM Model.}

Suppose we observe (some component of) $x(t) \in \CC^d$ from the solution of
\begin{align}
	\frac{d x}{dt} = A x, \quad x(0) = x_0,
\end{align}
where $A \in \{ A \in \CC^{d\times d} | A^* = -A\}$.  If $|x_0| =1$ then $|x(t)|$  can be interpreted as a discrete probability distribution on $\{1, \ldots, d\}$ for any $t \geq 0$.

\subsection{Estimation Problem 2: Back to the Advection Diffusion model.} 

Consider the advection-diffusion model
\begin{align}
	\partial_t \theta = (\kappa \Delta- \bfU \cdot \nabla) \theta \quad \theta(0) = \theta_0
\end{align}
Denote
\begin{align}
	S_\bfU(t) \theta_0 := \theta(t; \theta_0, \bfU).
\end{align}
Then taking 
\begin{align}
  \rho =  \nabla_\bfV S_\bfU(t) \theta_0 := \lim_{\epsilon \to 0} \epsilon^{-1} ( S_{\bfU+ \epsilon \bfV}(t) \theta_0 -  S_{\bfU}(t) \theta_0 )
\end{align}
we have the $\rho$ obeys
\begin{align}
\partial_t \rho = (\kappa \Delta- \bfU \cdot \nabla) +  \bfV \cdot \nabla \theta \quad \rho(0) = 0.
\end{align}
Here we still have the variation of constants formula
\begin{align}
	\rho(t) = \int_0^t S_\bfU(t-s) \bfV \cdot \nabla \theta(s) ds =  \int_0^t S_\bfU(t-s) \bfV \cdot \nabla S_\bfU(s) \theta_0 ds.
\end{align}
%Suppose we observe (some component of) $x(t) \in \CC^d$ from the solution of
%\begin{align}
%	\frac{d x}{dt} = A x \quad x(0) = x_0
%\end{align}
%where $A \in \{ A \in \CC^{d\times d} | A^* = -A\}$.  If $|x_0| =1$ then $|x(t)|$  can be interpreted as a discrete probability distribution on $\{1, \ldots, d\}$ for any $t \geq 0$.
%\section*{Acknowledgements}
%
%Our efforts are supported under the grants DMS-1816551, NSF-DMS-2108790 (NEGH) 
% NSF-DMS-2108791 (JAK), NSF-DMS-2108791 (AJH) and DMS-2009859 (CFM).





%% References

\begin{footnotesize}
\addcontentsline{toc}{section}{References}
\bibliographystyle{alpha}
\bibliography{refs}
\end{footnotesize}
 
%\bibliographystyle{sysbio}
%\bibliography{refs}
%
%
%
%\newpage
%\begin{multicols}{2}
%\noindent
%Nathan E. Glatt-Holtz\\ {\footnotesize
%Department of Mathematics\\
%Tulane University\\
%Web: \url{http://www.math.tulane.edu/~negh/}\\
%Email: \url{negh@tulane.edu}} \\[.2cm]
%
%
%\columnbreak
%
%\noindent Andrew J. Holbrook\\  {\footnotesize
%Department of Biostatistics\\
%University of California, Los Angeles\\
%Web: \url{https://andrewjholbrook.github.io}\\
%Email: \url{aholbroo@g.ucla.edu}}
%
%
% \end{multicols}



\end{document}

