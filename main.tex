% !TeX document-id = {d42653b9-9ea4-48c2-860b-3d6e6e9d8ab0}
% !TeX TXS-program:compile = txs:///pdflatex/[--shell-escape]
\documentclass[12pt]{article} % For LaTeX2e
%\usepackage{neurips_2021}
\usepackage[colorlinks, citecolor={blue}]{hyperref}
\usepackage{url}
\usepackage{amsfonts,amscd,amssymb}
\usepackage{amsthm,amsmath,natbib}
\usepackage{algorithm,algorithmicx,algpseudocode}
\usepackage{bm}
\usepackage{bbm} %bb font numbers
\usepackage[table]{xcolor}
\usepackage{verbatim}
\usepackage{graphicx}
\usepackage{setspace}
\usepackage{natbib}
\usepackage[margin=1in]{geometry}
\usepackage{enumitem}
%\usepackage[nolists]{endfloat}
\usepackage{listings}
\usepackage[textsize=tiny]{todonotes}
\usepackage{tikz}
\usetikzlibrary{shapes.misc}
\usepackage{etoolbox}
\usepackage{appendix}
\usepackage[format=plain,
labelfont={it},
textfont=it]{caption}
\usepackage{subcaption}
\usepackage{wrapfig}
\usepackage{xr}
\usepackage{booktabs}
\usepackage{multirow}
\usepackage{authblk}
\usepackage{mathbbol}
\usepackage{thmtools}



\usetikzlibrary{matrix}
\usetikzlibrary{backgrounds}
\usetikzlibrary{calc}
\usetikzlibrary{arrows,shapes}
\usetikzlibrary{decorations}
\usetikzlibrary{decorations.pathmorphing}
\usetikzlibrary{fit}
\usetikzlibrary{decorations.pathreplacing}
\usetikzlibrary{shapes.misc}
\usetikzlibrary{shapes.geometric}

\newtoggle{quickdraw}
%\toggletrue{quickdraw} % Uncomment this to render more quickly (non-random)


\definecolor{lightgrey}{rgb}{0.9,0.9,0.9}
\definecolor{darkgreen}{rgb}{0,0.3,0}
%\definecolor{darkred}{rgb}{0.3,0,0}

\definecolorset{rgb}{}{}{darkred,0.8,0,0;darkgreen,0,0.5,0;darkblue,0,0,0.5}

%\doublespacing

\newtheorem{thm}{Theorem}
\newtheorem{lemma}{Lemma}
\newtheorem{prop}{Proposition}
\newtheorem{cor}{Corollary}
\newtheorem{remark}{Remark}
\newtheorem{example}{Example}
\newtheorem{mydef}{Definition}
\newtheorem*{assumption}{Assumption}
\newtheorem{clm}{Claim}

\newcommand{\argmax}{\operatornamewithlimits{arg\,max}}
\newcommand{\argmin}{\operatornamewithlimits{arg\,min}}
\newcommand*{\fplus}{\genfrac{}{}{0pt}{}{}{+}}
\newcommand*{\fdots}{\genfrac{}{}{0pt}{}{}{\cdots}}
\newcommand{\mb}{\mathbf}
\newcommand{\mc}{\mathcal}
\newcommand{\dx}{\mbox{d}}

\renewcommand{\vec}[1]{\mathbf{#1}}
\newcommand{\numTaxa}{N}
\newcommand{\numTraits}{D}
\newcommand{\numDatasets}{M}
%\newcommand{\numLatent}{D}
\newcommand{\taxonIndex}{i}
\newcommand{\traitIndex}{j}
\newcommand{\traitData}{\vec{Y}}
\newcommand{\traitDatum}{y}
\newcommand{\datasetIndex}{m}
\newcommand{\exemplar}{\text{e}}

\newcommand{\sequences}{\vec{S}}
\newcommand{\latentData}{\vec{X}}
\newcommand{\latentdata}{\vec{x}}
\newcommand{\latentDatum}{x}
\newcommand{\phylogeneticParameters}{\boldsymbol{\phi}}
\newcommand{\phylogeny}{{\cal G}}
\newcommand{\tree}{\phylogeny}
%\newcommand{\otherParameters}{\boldsymbol{\
\newcommand{\transpose}{^{t}}

\newcommand{\distanceMatrix}{\mathbf{Y}}
\newcommand{\distance}{y}
\newcommand{\summant}{r}



\newcommand{\cdensity}[2]{\ensuremath{p(#1 \,|\,#2)}}
\newcommand{\density}[1]{\ensuremath{p(#1 )}}

\newcommand{\treeNode}{\nu}

\newcommand{\traitVariance}{\mathbf{\Sigma}}
\newcommand{\nodeIndex}{c}

%\newcommand{\parent}[1]{\mbox{\tiny pa}(#1)}
\newcommand{\parentBig}[1]{\mbox{pa}(#1)}

\newcommand{\sibling}[1]{\mbox{\tiny sib}(#1)}
\newcommand{\siblingBig}[1]{\mbox{sib}(#1)}

\newcommand{\rootMean}{\boldsymbol{\mu}_0}
\newcommand{\rootVarianceScalar}{\tau_0}
\newcommand{\unsequencedVarianceScalar}{\tau_{\exemplar}}
\newcommand{\treeVariance}{\vec{V}_{\tree}}
\newcommand{\hatTreeVariance}{\hat{\vec{V}}_{\tree}}
\newcommand{\mdsSD}{\sigma}
\newcommand{\mdsVariance}{\mdsSD^2}
\newcommand{\residual}{\hat{\traitDatum}}
\newcommand{\modelDistance}{\delta}
\newcommand{\cdf}{\phi}
\newcommand{\normalCDF}[1]{\Phi \left( #1 \right)}

\newcommand{\order}[1]{{\cal O}\hspace{-0.2em}\left( #1 \right)}

\newcommand{\rootNode}{\nu^{\datasetIndex}_{2 \numTaxa_{\datasetIndex} -1 }}
\newcommand{\pathLength}[1]{d(F, #1 )}
\newcommand{\pathLengthNew}[2]{
d_{F}
(
{#1}, {#2}
)
}
\newcommand{\J}{\vec{J}}
\newcommand{\pprime}{^{\prime}}
\newcommand{\otherIndex}{i \pprime}
\def\kronecker{\raisebox{1pt}{\ensuremath{\:\otimes\:}}}

\definecolor{trevorblue}{rgb}{0.330, 0.484, 0.828}
\definecolor{trevoryellow}{rgb}{0.829, 0.680, 0.306}


%\makeatletter
%\def\title@font{\Huge}
%\let\ltx@maketitle\@maketitle
%\def\@maketitle{\bgroup%
%	\let\ltx@title\@title%
%	\def\@title{\resizebox{\textwidth}{!}{%
%			\mbox{\title@font\ltx@title}%
%	}}%
%	\ltx@maketitle%
%	\egroup}
%\makeatother


\title{Matrix exponential derivative}
\date{}



\author{Andrew J.~Holbrook}


\affil{UCLA Biostatistics}





\renewcommand\Authands{ and }


\graphicspath{{figures/}}

\begin{document}


\maketitle




\begin{abstract}

[Abstract]


\end{abstract}



\section{Introduction}\label{sec:intro}

%\newcommand{\x}{\mathbf{x}}
%\renewcommand{\v}{\mathbf{v}}
%\newcommand{\TTheta}{\boldsymbol{\theta}}
%\newcommand{\TTTheta}{\boldsymbol{\Theta}}
%\newcommand{\tr}{\mbox{tr}}
%\newcommand{\X}{\mathbf{X}}
%\renewcommand{\u}{\mathbf{u}}
\newcommand{\QQ}{\mathbf{Q}}
\newcommand{\MM}{\mathbf{M}}
\newcommand{\JJ}{\mathbf{J}}
\newcommand{\II}{\mathbf{I}}
\newcommand{\RR}{\mathbf{R}}
%\newcommand{\haar}{\mathcal{H}}
%\newcommand{\orthog}{\mathcal{O}}
%\newcommand{\SSigma}{\boldsymbol{\Sigma}}
\newcommand{\Zero}{\boldsymbol{0}}


Let $\QQ=[q_{ij}]$ be the rate matrix for an $S$-dimensional continuous time Markov chain, where $q_{ij}>0$ for $i\neq j$, and $q_{ii} = -\sum_{j\neq i} q_{ij}$.  We are interested in quickly approximating the directional derivative of the matrix exponential
\begin{align*}
	\exp(t\QQ) = \sum_{k=0}^\infty \frac{t^k\QQ^k}{k!}
\end{align*}
for arbitrary $t>0$ using the formula \citep{najfeld1995derivatives}
\begin{align}\label{eq:deriv}
	\nabla_{\MM} \exp(t\QQ)  =  \exp(t\QQ)  \sum_{k=0}^\infty \frac{t^{k+1}}{(k+1)!} \{\MM,\QQ^k\} \, .
\end{align}
Here, $\MM$ gives the direction and $\{\MM,\QQ^k\}$ is the matrix commutator series satisfying the recursion
\begin{align*}
	\{\MM,\QQ^0\} &= \MM \, , \\ 
	\{\MM,\QQ^k\} &= [\{\MM,\QQ^{k-1}\} , \QQ]  = \{\MM,\QQ^{k-1}\} \QQ - \QQ \{\MM,\QQ^{k-1}\} \, .
\end{align*}
In particular, we are interested in the derivative with respect to all $S^2$ $q_{ij}$, and these collectively correspond to the derivative in the direction of the matrix
\begin{align*}
\JJ := \begin{bmatrix}
	1 & 1 &\cdots & 1 \\
1&	1 &   & \\
\vdots && \ddots& \\
	1 & &  & 1 
	\end{bmatrix}  \, .
\end{align*}
Letting $\Zero$ denote the zero matrix, the identities
\begin{align*}
\QQ\JJ&=\Zero \quad \mbox{and}\\
\exp(t\QQ) \JJ &= \JJ 
\end{align*}
lead to fact
\begin{align*}
 \exp(t\QQ)	\{\JJ , \QQ^1\} &=  \exp(t\QQ)	\left( \JJ  \QQ - \QQ \JJ     \right) \\
&= \JJ \QQ \, .
\end{align*} 
Further, if we assume $ \exp(t\QQ)	\{\JJ , \QQ^{k}\} = \JJ\QQ^{k} $ for $k\geq0$, we have
\begin{align*}
 \exp(t\QQ)	\{\JJ , \QQ^{k+1}\} &= \exp(t\QQ)  \left( 	\{\JJ , \QQ^{k}\} \QQ - \QQ 	\{\JJ , \QQ^{k}\} \right)  \\
 &= \JJ\QQ^{k} \QQ - \exp(t\QQ) \QQ 	\{\JJ , \QQ^{k}\} \\
 &= \JJ\QQ^{k+1}  - \QQ \exp(t\QQ)  \{\JJ , \QQ^{k}\} \\ 
 &= \JJ\QQ^{k+1} - \QQ \JJ \QQ^k \\
 &= \JJ \QQ^{k+1} \, ,
\end{align*}
where we use the fact that $\exp(t\QQ) \QQ = \QQ \exp(t\QQ)$. By an inductive argument, we may therefore write the following special case for $\eqref{eq:deriv}$:
\begin{align}\label{eq:deriv2}
	\nabla_{\JJ} \exp(t \QQ)  &=  \sum_{k=0}^\infty \frac{t^{k+1}}{(k+1)!} \JJ \QQ^k  = \JJ  \sum_{k=0}^\infty \frac{t^{k+1}}{(k+1)!}  \QQ^k \, .
\end{align}
This formula has an intuitive interpretation.   The derivative of the matrix power $\QQ^k$ in any direction $\MM$ is
\begin{align}\label{eq:power}
	\lim_{h\rightarrow 0}  \frac{(\QQ+h\MM)^k - \QQ^k }{h} = \sum_{p=1}^k  \QQ^{k-p} \MM \QQ^{p-1} \, ,
\end{align}
and the right-hand side reduces to $\JJ\QQ^{k-1}$ when we set $\MM=\JJ$.  Thus, Formula \eqref{eq:deriv2} naturally follows from the application of Equation \eqref{eq:power} to the individual terms within the series expansion of $\exp(t\QQ)$.

A second expansion due to \citet{najfeld1995derivatives} achieves faster convergence:
\begin{align}
	\nabla_{\MM} \exp(t\QQ)  =  t\exp(t\QQ/2)  \sum_{k=0}^\infty \frac{\{\MM,(t\QQ/2)^{2k}\} }{(2k+1)!} \exp(t\QQ/2) \, .
\end{align}
Similar arguments to those above provide the result
\begin{align}\label{eq:deriv3}
	\nabla_{\JJ} \exp(t\QQ)  =  t \sum_{k=0}^\infty  \frac{t^{2k}}{2^{2k}(2k+1)!} \JJ\QQ^{2k}\exp(t\QQ/2) \, .
\end{align}

\subsection{First-order approximations}

One may use Formula \eqref{eq:deriv2} to define the 1st-order approximation 
\begin{align*}
		\widetilde{\nabla}_{\JJ} \exp(t \QQ)  :=  t \JJ  
\end{align*}
and residual term
\begin{align*}
	\widetilde{\RR} := \JJ \sum_{k=1}^\infty \frac{t^{k+1}}{(k+1)!}  \QQ^k \, .
\end{align*}
Alternatively, one may use Formula  \eqref{eq:deriv3} to define the 1st-order approximation
\begin{align*}
		\widehat{\nabla}_{\JJ} \exp(t \QQ)  :=  t \JJ  \exp(t\QQ/2)
\end{align*}
and its respective residual
\begin{align*}
	\widehat{\RR} := t \sum_{k=1}^\infty  \frac{t^{2k}}{2^{2k}(2k+1)!} \JJ\QQ^{2k}\exp(t\QQ/2) \,.
\end{align*}



\bibliographystyle{sysbio}
\bibliography{refs}

%\section{Empirical accuracy of the extra-dimensional simplicial sampler}\label{sec:qqplots}
%
%We do not prove that the extra-dimensional sampler leaves the target distribution invariant, but limited simulations suggest it might.  Figure \ref{fig:accuracy} displays quantile-quantile plots for the sampler using a simplex with 101 vertices and targeting 3D Gaussian and mixture of two Gaussian distributions.  
%
%\begin{figure}[t!]
%	\centering
%	\includegraphics[width=\linewidth]{accuracyFig.png}
%	\caption{Quantile-quantile plots for the extra-dimensional sampler using a simplex with 101 vertices targeting a 3D Gaussian and mixture of two Gaussian distributions.}\label{fig:accuracy}
%\end{figure}

%\section{Empirically optimal acceptance rates for multiple-try Metropolis}\label{sec:mtmScaling}
%
%\citet{bedard2012scaling}
%
%\begin{figure}[t!]
%	\centering
%	\includegraphics[width=0.8\linewidth]{mtmScaling.pdf}
%	\caption{Optimal acceptance rates from 20 independent MCMC runs for each dimensionality of the a standard normal target}\label{fig:accuracy}
%\end{figure}


%%%%%%%%%%%%%%%%%%%%%%%%%%%%%%%%%%%%%%%%%%%%%%%%%%%%%%%%%%%%

\end{document}
